\documentclass{paper}
\usepackage[dutch]{babel}
\usepackage{vhistory}
\usepackage{graphicx}
\usepackage{float}
\usepackage{hyperref}
\usepackage{subcaption}
\usepackage{caption}
\usepackage{ragged2e}
\hypersetup{
    colorlinks=true,
    linkcolor=black,
    urlcolor=red,
    linktoc=all
}

\newcommand{\latest}{1.0}
\newcommand{\starrblock}[7]{\subsection{#1}{
    Beroepsproduct waar ik aan gewerkt heb: #2
    \paragraph{S}{
        #3
    }
    \paragraph{T}{
        #4
    }
    \paragraph{A}{
        #5
    }
    \paragraph{R}{
        #6
    }
    \paragraph{R}{
        #7
    }
}}

\title{STARR Reflectie: <Naam opdracht> \\
\large Studentnummer: <StudentNummer> \\
       E-Mail: <E-Mail>}
\author{<Author>}

\begin{document}
% Title page
\maketitle
\mbox{}
\vfill
\textbf{Versie: \latest}
\newpage

% Changes
\begin{versionhistory}
    \vhEntry{\latest}{<Date>}{<Author>}{<Notes>}
\end{versionhistory}
\newpage

% Contents  
\tableofcontents
\newpage

% STARRS
% Example command:  \starrblock{Deelcompetentie}{Product [X, Y, \dots]}{Situatie}{Taak}{Activiteit}{Resultaat}{Reflectie}
\section{Communicatief vermogen}{
  \starrblock{Deelcompetentie}{Product [X, Y, \dots]}{Situatie \textit{TEST}}{Taak}{Activiteit}{Resultaat}{Reflectie}
 }

\section{Professioneel vakmanschap}{

 }

\section{Onderzoekend vermogen}{

 }

\section{Leervermogen}{

 }

\section{Beroepsethiek en maatschappelijke oriëntatie}{

 }

\section{Samenwerken}{

 }
% Bijlagen (foto's etc)
\section{Bijlagen}{

 }

\end{document}

